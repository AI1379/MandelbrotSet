%! Author = Renatus Madrigal
%! Date = 3/6/2025

% Preamble
\documentclass[11pt]{article}

% Packages
\usepackage[
    left=0.8in,
    right=0.8in,
    top=0.8in,
    bottom=0.8in
]{geometry}
\usepackage[UTF8]{ctex}
\usepackage{amsmath}
\usepackage{amsfonts}
\usepackage{amssymb}
\usepackage[ruled,vlined]{algorithm2e}
\usepackage{listings}
\usepackage{color}
\usepackage{tikz}
\usepackage{hyperref}
\usepackage{graphicx}

\linespread{1.2}

\usetikzlibrary{arrows.meta, positioning, shapes.geometric}


\definecolor{codegreen}{rgb}{0,0.6,0}
\definecolor{codegray}{rgb}{0.5,0.5,0.5}
\definecolor{codepurple}{rgb}{0.58,0,0.82}
\definecolor{backcolour}{rgb}{0.95,0.95,0.92}

\lstset{
    frame=tb,
    language=C++,
    basicstyle={\small\ttfamily},
    backgroundcolor=\color{backcolour},
    commentstyle=\color{codegreen}\itshape,
    keywordstyle=\color{blue}\bfseries,
    numberstyle=\tiny\color{codegray}\ttfamily,
    stringstyle=\color{codepurple}\ttfamily,
    breakatwhitespace=false,
    breaklines=true,
    captionpos=b,
    keepspaces=true,
    numbers=left,
    numbersep=5pt,
    showspaces=false,
    showstringspaces=false,
    showtabs=false,
    tabsize=4
}

\newtheorem{definition}{定义}
\newtheorem{property}{性质}

\title{关于Mandelbrot集的报告}
\author{Renatus Madrigal}
\date{\today}
% Document
\begin{document}

    \maketitle


    \section{摘要}\label{sec:abstract}

    Mandelbrot 集是一个有趣的集合,具有自相似性和分形性。
    这个项目探索了 Mandelbrot 集和它的一些性质,并基于 C++ 和 OpenCV 提供了一个 Mandelbrot 集的可视化展示。
    我们将引入逃逸时间算法来生成 Mandelbrot 集,并使用 Sobel 算子做边界探测。
    我们还引入了一些优化,包括并行计算和创建异步流水线,以及一些精度上的优化。
    这个可视化项目展现了 Mandelbrot 集中独有的复杂与美妙。


    \section{背景}\label{sec:background}

    \begin{definition}
        令 $f_c(z) = z^2 + c$,其中 $z, c \in \mathbb{C}$。
        复数列 $z_n$ 满足:$z_0 = 0$ 且 $z_{n+1} = f_c(z_n)$。
        Mandelbrot 集是所有满足 $|z_n|$ 收敛的复数 $c$ 的集合。
        形式化地说,Mandelbrot 集定义为:
        \begin{equation}
            \label{eq:mandelbrot_set_define}
            M = \{c \in \mathbb{C} : \lim_{n \to \infty} |z_n| < \infty\}
        \end{equation}
    \end{definition}

    可以证明\textsuperscript{\cite{branner1989mandelbrot}},Mandelbrot 集具有以下性质:

    \begin{property}
        \label{property:bounded}
        $\forall c \in M,\ n \in \mathbb{N}_+ ,\ |z_n| \leq 2$
    \end{property}

    因此,我们可以定义\textbf{逃逸时间函数}:

    \begin{definition}
        令 $f_c(z) = z^2 + c$,其中 $z, c \in \mathbb{C}$。
        逃逸时间函数 $L_c(z)$ 定义为:
        \begin{equation}
            L(c) = \min\{n : z_n > 2\}\label{eq:layer_function}
        \end{equation}
    \end{definition}

    这个函数衡量了序列 $\{z_n\}$ 发散的速率。


    \section{算法}\label{sec:algorithm}

    \subsection{逃逸时间算法}\label{subsec:escape_time_algorithm}

    为了生成 Mandelbrot 集,我们使用\textbf{逃逸时间算法}。

    \begin{algorithm}[H]
        \SetAlgoLined
        \caption{Escape Time Algorithm}
        \label{alg:escape_time_algorithm}
        \KwData{Complex number $c$, maximum number of iterations $N$}
        \KwResult{Number of iterations $n$}
        z = 0\;
        \While{$|z| \leq 2$ and $n < N$}{
            $z \leftarrow z^2 + c$\;
            $n \leftarrow n + 1$\;
        }
        \Return $n$\;
    \end{algorithm}

    迭代阈值 $N$ 通常设定为 $N = 1000$,从而在计算效率和视觉效果之间取得平衡。

    \subsection{边界探测}\label{subsec:boundary_detection}

    为了更好地分析 Mandelbrot 集的结构,我们引入了一种基于逃逸时间函数的梯度场的边界探测方法。
    具体来说,我们使用 Sobel 算子\textsuperscript{\cite{sobel1968isotropic}}来计算离散梯度的近似值。

    Sobel 算子定义如下:

    \begin{equation}
        \label{eq:sobel_operator}
        G_x = \begin{bmatrix}
                  -1 & 0 & 1 \\
                  -2 & 0 & 2 \\
                  -1 & 0 & 1
        \end{bmatrix}
        \quad
        G_y = \begin{bmatrix}
                  -1 & -2 & -1 \\
                  0  & 0  & 0  \\
                  1  & 2  & 1
        \end{bmatrix}
    \end{equation}

    然后,梯度幅度计算如下:

    \begin{equation}
        \label{eq:gradient_magnitude}
        \parallel \nabla L \parallel \ = \sqrt{(G_x * L)^2 + (G_y * L)^2}
    \end{equation}

    然而,在实际中,我们通常取两个分量的算术平均值作为梯度幅度。
    这种简化使得计算更加高效。

    接着,我们将 \textit{GradientThreshold} 设为 0.5,选出梯度幅度高于这个阈值的像素点。
    这些像素点比较靠近 Mandelbrot 集的边界,因此我们可以在这些区域放大,从而展示更多的细节。

    得到 mask 后,我们可以探测其中梯度幅度较高的像素点,他们极有可能在 Mandelbrot 集的分形边界上。
    具体而言,我们将整个矩阵划分为不同的块,对于每一个块,我们计算其中梯度幅度大于阈值的点的个数。
    然后,我们可以选取这个值最大的块,放大这个块,从而展示 Mandelbrot 集的分形特征。
    以下是伪代码:

    \begin{algorithm}[H]
        \SetAlgoLined
        \caption{Detect Boundary}
        \label{alg:detect_boundary}
        \KwResult{Boundary area}
        $mask,\ maxHighGradientPixels,\ maxBlock \leftarrow detectHighGradient(matrix),\ 0,\ null$\;
        Split matrix into $n \times n$ blocks\;
        \For{each block}{
            Count the number of high gradient pixels\;
            \If{high gradient pixels > maxHighGradientPixels}{
                $maxHighGradientPixels,\ maxBlock \leftarrow high gradient pixels,\ block$\;
            }
        }
        \Return maxBlock\;
    \end{algorithm}

    我们把最大的一块取出来,然后将它的中心设置为新的中心。
    然后,放大这块区域,重新生成图像,从而获得更多的细节。
    为了实现这个,我们需要调整图像的缩放和中心。
    OpenCV 提供了一些相关函数来生成缩放矩阵,并且可以直接使用 \texttt{cv::warpAffine} 函数来实现缩放。
    这些代码在
    \href{https://github.com/AI1379/MandelbrotSet/blob/master/src/VideoGenerator.h}{\texttt{src/VideoGenerator.h}} 中。


    \section{优化}\label{sec:optimizations}

    为了优化运行效率和精度,我们引入了一些优化方法。

    \subsection{并行计算}\label{subsec:parallelism}

    因为 Mandelbrot 集中每一个像素都是独立的,因此我们可以并行计算每一个像素的逃逸时间。
    我们引入了 OpenMP 或者 CUDA 来实现并行计算。

    在 CPU 版本的实现中,我们使用 OpenMP 来并行计算每一个像素的逃逸时间。
    由于现代编译器已经集成了 OpenMP 的支持,我们可以很容易地实现并行计算,如下所示:

    \begin{lstlisting}[label={lst:openmp_parallel_for}, gobble=8]
        #pragma omp parallel for
    \end{lstlisting}

    在 GPU 版本的实现中,我们使用 CUDA 来加速逃逸时间的计算。
    我们将图像划分为不同的块,然后为每一个块分配一个 CUDA 线程。
    这样,我们可以并行计算每一个像素的逃逸时间。
    这些代码可以在 \href{https://github.com/AI1379/MandelbrotSet/blob/master/src/MandelbrotSetCuda.cu}
    {\texttt{src/MandelbrotSetCuda.cu}} 中找到。

    \subsection{异步流水线}\label{subsec:asynchronous-pipeline}

    在视频生成中,我们需要构建一个更复杂的异步流水线。
    因此,我们引入了 \texttt{std::execution}\textsuperscript{\cite{P2300Proposal}} 来构建异步流水线。
    P2300 提案提供了一个高层次的抽象,是 C++ 标准异步模型的一个候选,极有可能进入 C++26 标准。
    基于此,我们设计了这样的一个异步流水线:

    \vspace{0.1cm}

    \begin{center}
        \begin{tikzpicture}[
            node distance=1.5cm,
            startstop/.style={
                rectangle,
                rounded corners,
                minimum width=2cm,
                minimum height=0.8cm,
                text centered,
                draw=black,
                fill=red!30
            },
            process/.style={
                rectangle,
                minimum width=2cm,
                minimum height=0.8cm,
                text centered,
                draw=black,
                fill=blue!30
            },
            decision/.style={
                diamond,
                aspect=3,
                minimum width=1cm,
                minimum height=1cm,
                text centered,
                draw=black,
                fill=green!30
            },
            arrow/.style={
                thick,
                ->,
                >=Stealth[round]
            }
        ]
            \node (start) [startstop] {开始};
            \node (coro) [process, below=0.8cm of start] {启动协程};
            \node (generate) [process, right=1.2cm of start] {生成关键帧};
            \node (detect) [process, right=1.2cm of generate] {检测边界};
            \node (send) [process, right=1.5cm of detect] {发送关键帧};
            \node (stopsignal) [process, below=1.6cm of send] {发送停止信号};
            \node (zoom) [process, right=1.2cm of coro] {放大};
            \node (wait) [process, right=1.2cm of zoom] {等待下一帧};
            \node (waitsignal) [decision, below=0.8cm of wait] {是否受到信号?};
            \node (waitqueue) [process, below=0.8cm of waitsignal] {等待队列中的任务完成};
            \node (waitall) [process, below right=0.5cm and -2cm of waitqueue] {等待所有任务完成};
            \node (end) [startstop, below=0.5cm of waitall] {结束};

            \draw [arrow] (start) -- (generate);
            \draw [arrow] (generate) -- (detect);
            \draw [arrow] (detect) -- (send);
            \draw [->] (send) -- node[midway] {生成下一帧} ++(0, 1) -| (generate);
            \draw [arrow] (start) -- (coro);
            \draw [arrow] (coro) -- (zoom);
            \draw [arrow] (zoom) -- (wait);
            \draw [<-, dashed] (wait) --node {获取关键帧} (send);
            \draw [arrow] (wait) -- (waitsignal);
            \draw [->] (waitsignal) -- node[left] {是} (waitqueue);
            \draw [->] (waitsignal) -- node[above] {否} ++(-3, 0) -| (zoom);
            \draw [arrow] (send) -- (stopsignal);
            \draw [arrow] (waitqueue) -- (waitall);
            \draw [arrow] (stopsignal) -- (waitall);
            \draw [arrow] (waitall) -- (end);

        \end{tikzpicture}
    \end{center}

    依靠 P2300 提案提供的工具,我们可以很容易地构建这样一个异步流水线,从而实现高性能的视频生成。
    这些代码在 \href{https://github.com/AI1379/MandelbrotSet/blob/master/src/VideoGenerator.h}
    {\texttt{src/VideoGenerator.h}} 中。

    \subsection{精度优化}\label{subsec:precision-optimization}

    为了提高计算精度,我们引入了\texttt{ExtendedDouble} 类型来存储更多的有效数字。
    这个类通过一个额外的\texttt{int}来存储有效数字的位数,从而提高了计算的精度。
    这在高精度计算中非常有用,同时它的性能可以接收,且远高于\texttt{mpfr\_t}。
    具体的实现参见\href{https://github.com/AI1379/MandelbrotSet/blob/master/src/ExtendedDouble.cu}
    {\texttt{src/ExtendedDouble.cu}}。


    \section{结论}\label{sec:conclusion}

    在这个项目中,我们探索了 Mandelbrot 集及其性质,并基于 OpenCV 和 CUDA 提供了一个 Mandelbrot 集的可视化展示。
    我们引入了逃逸时间算法来生成 Mandelbrot 集,以及 Sobel 算子用于边界探测。
    我们还优化了生成过程,包括并行计算、异步流水线和精度优化。
    这些优化显著提高了 Mandelbrot 集生成的效率和精度。

    最后,我们成功生成了 Mandelbrot 集的高保真度可视化,以及对应的视频。
    由于我们采用了随机的颜色方案,因此每一张生成的图片都是独一无二的,具有较好的视觉效果。
    以下是一个生成的图片的例子:

    \begin{figure}[htbp]
        \label{fig:figure}
        \centering
        \includegraphics[width=0.4\textwidth]{MandelbrotSet}
        \caption{Mandelbrot set}
    \end{figure}

    \bibliography{references}
    \bibliographystyle{plain}
\end{document}