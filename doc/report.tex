%! Author = Renatus Madrigal
%! Date = 3/6/2025

% Preamble
\documentclass[11pt]{article}

% Packages
\usepackage{amsmath}
\usepackage{amsfonts}
\usepackage{amssymb}
\usepackage[ruled,vlined]{algorithm2e}

\newtheorem{definition}{Definition}
\newtheorem{property}{Property}

\title{A report on the Mandelbrot set}
\author{Renatus Madrigal}
\date{\today}

% Document
\begin{document}

    \maketitle


    \section{Abstract}\label{sec:abstract}
    % TODO: Write abstract


    \section{Introduction}\label{sec:introduction}

    The Mandelbrot set constitutes a fractal structure characterized by an elegant mathematical definition that belies
    its extraordinary complexity.
    This project aims to explore the Mandelbrot set and its properties, as well as to provide a visual representation of
    the set using C++.
    This report will detail the algorithms and optimizations used to generate the Mandelbrot set.


    \section{Background}\label{sec:background}

    \begin{definition}
        Let $f_c(z) = z^2 + c$, where $z, c \in \mathbb{C}$.
        Sequence $\{z_n\}$ is defined by $z_0 = 0$ and $z_{n+1} = f_c(z_n)$.
        The \textbf{Mandelbrot set} $M$ is defined as follows:
        \begin{equation}
            M = \{c \in \mathbb{C} : \lim_{n \to \infty} |z_n| < \infty\}\label{eq:mandelbrot_set_define}
        \end{equation}
    \end{definition}

    It can be proved\textsuperscript{\cite{branner1989mandelbrot}} that the Mandelbrot set has this property:

    \begin{property}
        \label{prop:bounded}
        Let $c \in M$.
        Then, $|z_n| \leq 2$ for all $n$.
    \end{property}

    Thus, we can define the \textbf{layer function}:

    \begin{definition}
        Let $f_c(z) = z^2 + c$.
        The \textbf{layer function} $L_c(n)$ is defined as follows:
        \begin{equation}
            L(c) = \min\{n : z_n > 2\}\label{eq:layer_function}
        \end{equation}
    \end{definition}

    This function measures the escape speed of the sequence $\{z_n\}$.
    Then, we can define the \textbf{layer set} as follows:

    \begin{definition}
        \begin{equation}
            L_n = \{c \in \mathbb{C} : L(c) = n \} \label{eq:layer_set}
        \end{equation}
    \end{definition}

    We assert that for all $n$, $L_n$ is a connected set.
    With these definitions and properties, we can generate the image of the Mandelbrot set and explore its properties.


    \section{Algorithms}\label{sec:algorithms}

    To generate the Mandelbrot set, we use the \textbf{Escape Time Algorithm}.

    \begin{algorithm}[H]
        \SetAlgoLined
        \caption{Escape Time Algorithm}
        \label{alg:escape_time_algorithm}
        \KwData{Complex number $c$, maximum number of iterations $N$}
        \KwResult{Number of iterations $n$}
        z = 0\;
        \While{$|z| \leq 2$ and $n < N$}{
            $z \leftarrow z^2 + c$\;
            $n \leftarrow n + 1$\;
        }
        \Return $n$\;
    \end{algorithm}

    The maximum number of iterations $N$ is usually set to 1000.
    We will use this algorithm to generate the Mandelbrot set.


    \section{Optimizations}\label{sec:optimizations}





    \bibliography{references}
    \bibliographystyle{plain}
\end{document}